\documentclass{article}
\usepackage[utf8]{inputenc}
\documentclass{article}
\usepackage{listings}
\usepackage{graphicx}
\usepackage{epstopdf}
\usepackage{verbatim}
\usepackage[english]{babel}

\title{Labwork1 Avanced programming HPC}
\author{Do Quang Huy}
\date{October 2018}
\lstset{breaklines=true} 

\begin{document}
\maketitle

\section{Parallel Computing}
I used the extractly the same code from labwork\_CPU function for the to complete the labwork\_OpenMP, exclude that using one more line of code before the for loop to change from sequence computing to parallel computing 

\begin{verbatim}
#pragma omp parallel for
\end{verbatim}

the computation time when applying the parallel computing is 4 times faster than the sequence computing.  

\subsection{Measure time for the speedup}
measure speedup
\begin{verbatim}
USTH ICT Master 2018, Advanced Programming for HPC.
Warming up...
Starting labwork 1
labwork 1 CPU ellapsed 3560.3ms
labwork 1 OpenMP ellapsed 802.6ms
\end{verbatim}


for static schedule
\begin{verbatim}
USTH ICT Master 2018, Advanced Programming for HPC.
Warming up...
Starting labwork 1
labwork 1 CPU ellapsed 3552.1ms
labwork 1 OpenMP ellapsed 737.5ms
\end{verbatim}

for dynamic schedule
\begin{verbatim}
USTH ICT Master 2018, Advanced Programming for HPC.
Warming up...
Starting labwork 1
labwork 1 CPU ellapsed 3580.2ms
labwork 1 OpenMP ellapsed 701.1ms
\end{verbatim}

\end{document}
